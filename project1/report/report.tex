\documentclass[a4paper,titlepage]{article}
\usepackage[utf8]{inputenc} %Make sure all UTF8 characters work in the document
\usepackage{listings} %Add code sections
\usepackage{color}
\usepackage{graphicx}
\usepackage{titling}
\usepackage{textcomp}
\usepackage[hyphens]{url}
\usepackage[bottom]{footmisc}
\definecolor{listinggray}{gray}{0.9}
\definecolor{lbcolor}{rgb}{0.9,0.9,0.9}
\lstset{	%ifall du ska koda.
		backgroundcolor=\color{lbcolor},
		tabsize=4,
		rulecolor=,
		upquote=true,
		aboveskip={1.5\baselineskip},
		columns=fixed,
		showstringspaces=false,
		extendedchars=true,
	    breaklines=true,
	    prebreak = \raisebox{0ex}[0ex][0ex]{\ensuremath{\hookleftarrow}},
	    frame=single,
	    showtabs=false,
	    showspaces=false,
	    showstringspaces=false,
	    identifierstyle=\ttfamily,
	    keywordstyle=\color[rgb]{0,0,1},
	    commentstyle=\color[rgb]{0.133,0.545,0.133},
	    stringstyle=\color[rgb]{0.627,0.126,0.941},
}

%Set page size
\usepackage{geometry}
\geometry{margin=3cm}
\usepackage{parskip} 
%\pretitle{%	en bild för framsidan
	%\begin{center}
	%\LARGE
%	\includegraphics[width=6cm]{python.png}\\[\bigskipamount]
%}
\title{
    \textbf{Miniprojekt 1 -- TANA21}}
\date{\today}
\author{%
    Malcolm Vigren \\
    \texttt{malvi108@student.liu.se}
    \and
    Frans Skarman\\
    \texttt{frask812@student.liu.se}
    }
\renewcommand*\contentsname{Innehållsförteckning}

\begin{document}
\maketitle
\newpage
\tableofcontents
\newpage

\section{Introduktion}
Numerisk integration är en viktig del i många beräkningar. Det är ofta svårt
eller omöjligt att integrera godtyckliga funktioner och datamängder analytiskt,
särskilt på datorer. Numerisk integration är dock inte exakt i det generella
fallet, vilket innebär att approximationer måste göras. En av dessa metoder
är trapetsmetoden, som approximerar integralen med hjälp av förstagradspolynom.
Denna studie siktar på att utvärdera hur väl denna metod presterar,
både noggrannhetsmässigt och beräkningskomplexitetsmässigt.

\subsection{Syfte}
Syftet med denna undersökning är att utvärdera trapetsmetoden för numerisk
beräkning av integraler. Noggrannhetsordningen samt den aritmetiska
komplexiteten hos metoden ska utvärderas.

\section{Teori}
Trapetsregeln approximerar integralen genom att anpassa förstagradspolynom till
funktionen som ska integreras, och beräkna arean under dessa. Detta kan
beräknas med formeln

\begin{equation}
    \int_{x_0}^{x_n}f(x)dx \approx T(h) = \frac{h}{2}(f(x_0) +
    2\sum_{k=1}^{n-1}f(x_k) + f(x_n))
\end{equation}

där $h$ är en steglängd.

\subsection{Experimentbeskrivning}

För att utvärdera noggranhetsordningen utvärderades ett antal integraler
analytiskt. Sedan beräknades det numeriska värdet av samma integraler med
trapetsmetoden och olika steglängder.

% TODO: Bättre namn än delta x
För att beräkna felet vid varje steglängd $\delta x$ beräknades differensen mellan de numeriskt
bestämda värdena och de analytiskt bestämda värdena.

% TODO: Skriv om hur vi får exponenten

På ett liknande sätt beräknades metodens aritmetiska komplexitet. Tiden det tog att köra
algoritmen vid olika steglängder bestämdes genom att utföra 1000 iterationer av beräkningen
och mäta tiden det tog.

\section{Resultat}

\begin{table}[h]
    \begin{tabular}{l | l | l | l | l | l}
        \textbf{Exp} & \textbf{1-degree} & \textbf{2-degree} &
        \textbf{3-degree} & \textbf{atan-shit} & \textbf{periodic} \\ \hline
    0.0014316629 & 0.0000000000 & 0.0016666667 & 0.0401016661 & -0.0016666647 & 0.0000000000 \\
    0.0000143190 & -0.000000000 & 0.0000166667 & 0.0008201506 & -0.0000166667 & 0.0000000000 \\
    0.0000001432 & 0.0000000000 & 0.0000001667 & 0.0000083320 & -0.0000001667 & 0.0000000000 \\
    0.0000000014 & -0.000000000 & 0.0000000017 & 0.0000000833 & -0.0000000017 & 0.0000000000 \\
    0.0000000000 & 0.0000000000 & 0.0000000000 & 0.0000000008 & -0.0000000000 & 0.0000000000 \\
    \end{tabular}
\end{table}

\end{document}
