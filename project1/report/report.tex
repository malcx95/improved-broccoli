\documentclass[a4paper,titlepage]{article}
\usepackage[utf8]{inputenc} %Make sure all UTF8 characters work in the document
\usepackage{listings} %Add code sections
\usepackage{color}
\usepackage{graphicx}
\usepackage{titling}
\usepackage{textcomp}
\usepackage[hyphens]{url}
\usepackage[bottom]{footmisc}
\definecolor{listinggray}{gray}{0.9}
\definecolor{lbcolor}{rgb}{0.9,0.9,0.9}
\lstset{	%ifall du ska koda.
		backgroundcolor=\color{lbcolor},
		tabsize=4,
		rulecolor=,
		upquote=true,
		aboveskip={1.5\baselineskip},
		columns=fixed,
		showstringspaces=false,
		extendedchars=true,
	    breaklines=true,
	    prebreak = \raisebox{0ex}[0ex][0ex]{\ensuremath{\hookleftarrow}},
	    frame=single,
	    showtabs=false,
	    showspaces=false,
	    showstringspaces=false,
	    identifierstyle=\ttfamily,
	    keywordstyle=\color[rgb]{0,0,1},
	    commentstyle=\color[rgb]{0.133,0.545,0.133},
	    stringstyle=\color[rgb]{0.627,0.126,0.941},
}

%Set page size
\usepackage{geometry}
\geometry{margin=3cm}
\usepackage{parskip} 
%\pretitle{%	en bild för framsidan
	%\begin{center}
	%\LARGE
%	\includegraphics[width=6cm]{python.png}\\[\bigskipamount]
%}
\title{
    \textbf{Miniprojekt 1 -- TANA21}}
\date{\today}
\author{%
    Malcolm Vigren \\
    \texttt{malvi108@student.liu.se}
    \and
    Frans Skarman\\
    \texttt{frask812@student.liu.se}
    }
\renewcommand*\contentsname{Innehållsförteckning}
\renewcommand*\tablename{Tabell}

\begin{document}
\maketitle
\newpage
\tableofcontents
\newpage

\section{Introduktion}
Numerisk integration är en viktig del i många beräkningar. Det är ofta svårt
eller omöjligt att integrera godtyckliga funktioner och datamängder analytiskt,
särskilt på datorer. Numerisk integration är dock inte exakt i det generella
fallet, vilket innebär att approximationer måste göras. En av dessa metoder
är trapetsmetoden, som approximerar integralen med hjälp av förstagradspolynom.
Denna studie siktar på att utvärdera hur väl denna metod presterar,
både noggrannhetsmässigt och beräkningskomplexitetsmässigt.

\subsection{Syfte}
Syftet med denna undersökning är att utvärdera trapetsmetoden för numerisk
beräkning av integraler. Noggrannhetsordningen samt den aritmetiska
komplexiteten hos metoden ska utvärderas.

\section{Teori}
Trapetsregeln approximerar integralen genom att anpassa förstagradspolynom till
funktionen som ska integreras, och beräkna arean under dessa. Detta kan
beräknas med formeln

\begin{equation}
    \int_{x_0}^{x_n}f(x)dx \approx T(h) +
    2\sum_{k=1}^{n-1}f(x_k) + f(x_n))
\end{equation}

där $h$ är en steglängd.

\subsection{Experimentbeskrivning}

% Utvärderades utvärderades utvärderades
För att utvärdera noggranhetsordningen utvärderades ett antal integraler
analytiskt. Sedan beräknades det numeriska värdet av samma integraler med
trapetsmetoden och olika steglängder. För beräkningen av noggranhetsordning
användes steglängderna $h_i = \frac{1}{10^i}, i \in {1,2, \dots 5}$.
För att beräkna felet vid varje steglängd $h_i$ beräknades differensen mellan de numeriskt
bestämda värdena och de analytiskt bestämda värdena.

För att sedan beräkna noggranhetsordningen användes ekvation \ref{eq:orderofaccuracy} där $R_t$
är felet som funktion av steglängden och $c$ är en konstant

\begin{equation}
    \label{eq:orderofaccuracy}
    \frac{| R_t(h_{i+1})|}{| R_t(h_{i})|}  = \frac{ch_{i+1}^p}{ch_{i}^p} = 10^p
    \Leftrightarrow p \approx \log_{10}\left( \frac{h_{i+1}}{h_i} \right)
\end{equation}

Följande funktioner användes för evaluering av noggranhetsordningen:

\begin{itemize}
    \item Exponentialfunktionen $e^x$ med primitiv funktion $e^x$, under intervallet $[0, 1]$
    \item Förstagradspolynomet $x + 1$ med primitiv funktion $\frac{x^2}{2} + x$, under intervallet $[0, 1]$
    \item Andragradspolynomet $x^2 + 2x + 1$ med primitiv funktion
        $\frac{x^3}{3} + x^2 + x$, under intervallet $[0, 1]$
    \item 100-gradspolynomet $x^{100}$ med primitiv funktion
        $\frac{x^{101}}{101}$, under intervallet $[0, 1]$
    \item Funktionen $\frac{4}{1 + x^2}$ med primitiv funktion $4\arctan(x)$, under intervallet $[0, 1]$
    \item Den periodiska funktionen $\sin^2(x)$ med primitiv funktion
        $\frac{1}{4}\sin(2x)$, under intervallet $[0, \pi]$
\end{itemize}

% TODO: Skriv om hur vi får exponenten

På ett liknande sätt beräknades metodens aritmetiska komplexitet. Tiden det tog att köra
algoritmen vid olika steglängder bestämdes genom att utföra 100 iterationer av beräkningen
och mäta tiden det tog. Precis som för noggranhetsordningen kan aritmetiska komplexiteten
beräknas med Ekvation \ref{eq:arithmetic_complexity} där $T$ betecknar exekveringstiden
vid en viss steglängd. Här användes $h_i = \frac{1}{2^i}, i \in {20,21, \dots 25}$

\begin{equation}
    \label{eq:arithmetic_complexity}
    \frac{| T(h_{i+1})|}{| T_t(h_{i})|}  = \frac{ch_{i+1}^p}{ch_{i}^p} = 2^p
    \Leftrightarrow p \approx \log_{2}\left( \frac{h_{i+1}}{h_i} \right)
\end{equation}

Följande funktioner användes för evaluering av noggranhetsordningen:

\begin{itemize}
    \item Exponentialfunktionen $e^x$
    \item Andragradspolynomet $x^2 + 2x + 1$
    \item Funktionen $\frac{4}{1 + x^2}$
\end{itemize}

\section{Resultat}


exp
potens: 2.00

1-degree
potens: Inf


2-degree
potens: 2.00


3-degree
potens: 1.94


atan shit
potens: 2.00


periodic
potens: -0.25




\subsection{Noggranhetsordning}

\begin{table}[h]
    \begin{tabular}{l | r | r | r | r | r | r}
        $\mathbf{1/h}$ & $\mathbf{e^x}$ & $\mathbf{x + 1}$ & $\mathbf{x^2 + 2x + 1}$ & $\mathbf{x^{100}}$ & $\mathbf{\frac{4}{1 + x^2}}$ & $\mathbf{\sin^2(x)}$ \\ \hline
        $10^1$         & $1.432 \cdot 10^{-03}$ & $0                    $  & $1.667 \cdot 10^{3}  $ & $4.010 \cdot 10^{-2} $ & $-1.667 \cdot 10^{-03}$ & $2.220 \cdot 10^{-16}$ \\
        $10^2$         & $1.432 \cdot 10^{-05}$ & $-2.220 \cdot 10^{-16}$  & $1.667 \cdot 10^{-5} $ & $8.202 \cdot 10^{-4} $ & $-1.667 \cdot 10^{-05}$ & $4.441 \cdot 10^{-16}$ \\
        $10^3$         & $1.432 \cdot 10^{-07}$ & $0                    $  & $1.667 \cdot 10^{7}  $ & $8.332 \cdot 10^{-6} $ & $-1.667 \cdot 10^{-07}$ & $6.661 \cdot 10^{-16}$ \\
        $10^4$         & $1.432 \cdot 10^{-09}$ & $-4.441 \cdot 10^{-16}$  & $1.667 \cdot 10^{-9} $ & $8.333 \cdot 10^{-8} $ & $-1.667 \cdot 10^{-09}$ & $7.772 \cdot 10^{-15}$ \\
        $10^5$         & $1.432 \cdot 10^{-11}$ & $2.220e \cdot 10^{-16}$  & $1.666 \cdot 10^{-11}$ & $8.333 \cdot 10^{-10}$ & $-1.668 \cdot 10^{-11}$ & $7.994 \cdot 10^{-15}$ \\ \hline
        \textbf{p}     & 2.00                   & Inf                      & 2.00                   & 1.94                   & 2.00                    & -0.25 \\
    \end{tabular}
    \caption{Fel vid för olika funktioner vid olika steglängder}
    \label{tab:errors}
\end{table}

Tabell \ref{tab:errors}

\subsection{Aritmetisk komplexitet}



\begin{table}[h]
    \begin{tabular}{l | l | l}
        \textbf{Exp} & \textbf{2-degree} & \textbf{atan-shit} \\ \hline

    \caption{The results of the test of arithmetic complexity}
    \end{tabular}
\end{table}


\section{Diskussion}

De erhållna resultaten stämmer mestadels överens med vad som förväntades. De integraler
som inte har fel ger noggranhetsordning som ungefär är lika med 2 vilket stämmer överrens
med teorin.

% TODO: Utveckla och eller omformulera
Även aritmetiska komplexiteten stämmer väl överrens med det teorin som säger att
den borde vara 1. Detta stämmer dock inte för större steglängder där overhead från
for-loopar ger ett större relativt bidrag.

\end{document}
