\documentclass[a4paper,titlepage]{article}
\usepackage[utf8]{inputenc} %Make sure all UTF8 characters work in the document
\usepackage{listings} %Add code sections
\usepackage{color}
\usepackage{graphicx}
\usepackage{titling}
\usepackage{textcomp}
\usepackage[hyphens]{url}
\usepackage[bottom]{footmisc}
\definecolor{listinggray}{gray}{0.9}
\definecolor{lbcolor}{rgb}{0.9,0.9,0.9}
\lstset{	%ifall du ska koda.
		backgroundcolor=\color{lbcolor},
		tabsize=4,
		rulecolor=,
		upquote=true,
		aboveskip={1.5\baselineskip},
		columns=fixed,
		showstringspaces=false,
		extendedchars=true,
	    breaklines=true,
	    prebreak = \raisebox{0ex}[0ex][0ex]{\ensuremath{\hookleftarrow}},
	    frame=single,
	    showtabs=false,
	    showspaces=false,
	    showstringspaces=false,
	    identifierstyle=\ttfamily,
	    keywordstyle=\color[rgb]{0,0,1},
	    commentstyle=\color[rgb]{0.133,0.545,0.133},
	    stringstyle=\color[rgb]{0.627,0.126,0.941},
}

%Set page size
\usepackage{geometry}
\geometry{margin=3cm}
\usepackage{parskip} 
%\pretitle{%	en bild för framsidan
	%\begin{center}
	%\LARGE
%	\includegraphics[width=6cm]{python.png}\\[\bigskipamount]
%}
\title{
    \textbf{Miniprojekt 1 -- TANA21}}
\date{\today}
\author{%
    Malcolm Vigren \\
    \texttt{malvi108@student.liu.se}
    \and
    Frans Skarman\\
    \texttt{frask812@student.liu.se}
    }
\renewcommand*\contentsname{Innehållsförteckning}
\begin{document}
	\maketitle
	\newpage
\tableofcontents
\newpage

\section{Introduktion}
Numerisk integration är en viktig del i många beräkningar. Det är ofta svårt
eller omöjligt att integrera godtyckliga funktioner och datamängder analytiskt,
särskilt på datorer. Numerisk integration är dock inte exakt i det generella
fallet, vilket innebär att approximationer måste göras. En av dessa metoder
är trapetsmetoden, som approximerar integralen med hjälp av förstagradspolynom.
Denna studie siktar på att utvärdera hur väl denna metod presterar,
både noggrannhetsmässigt och beräkningskomplexitetsmässigt.

\subsection{Syfte}
Syftet med denna undersökning är att utvärdera trapetsmetoden för numerisk
beräkning av integraler. Noggrannhetsordningen samt den aritmetiska
komplexitetens 

\end{document}
