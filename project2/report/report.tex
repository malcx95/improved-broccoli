\documentclass[a4paper,titlepage]{article}
\usepackage[utf8]{inputenc} %Make sure all UTF8 characters work in the document
\usepackage{listings} %Add code sections
\usepackage{color}
\usepackage{graphicx}
\usepackage{titling}
\usepackage{textcomp}
\usepackage{amsmath}
\usepackage[hyphens]{url}
\usepackage[bottom]{footmisc}
\definecolor{listinggray}{gray}{0.9}
\definecolor{lbcolor}{rgb}{0.9,0.9,0.9}
\usepackage[framed,numbered,autolinebreaks,useliterate]{mcode}

%Set page size
\usepackage{geometry}
\geometry{margin=3cm}
\usepackage{parskip} 
%\pretitle{%	en bild för framsidan
	%\begin{center}
	%\LARGE
%	\includegraphics[width=6cm]{python.png}\\[\bigskipamount]
%}
\title{%
    \textbf{Projekt 2 i Beräkningsmatematik -- Icke-linjära ekvationer }}
\date{\today}
\author{%
    Malcolm Vigren \\
    \texttt{malvi108@student.liu.se}
    \and
    Frans Skarman\\
    \texttt{frask812@student.liu.se}
    }
\renewcommand*\contentsname{Innehållsförteckning}
\renewcommand*\tablename{Tabell}

\begin{document}
\maketitle
\newpage
\tableofcontents
\newpage

\section{Introduktion}

Även om en analyskurs på universitetet får det att verka som att man kan hitta rötter
och derivator till alla möjliga funktioner är så inte fallet, ibland är funktionerna
mer komplexa än vad våra verktyg klarar av. Då måste man istället använda
numeriska metoder för att hitta rötter. Sekantmetoden, som beskrivs i denna
rapport, är en sådan metod.

\subsection{Uppgift}

Syftet med denna undersökning är att implementera och utvärdera sekantmetoden
för lösning av ickelinjära ekvationer. Konvergensordningen av metoden vid
beräkning av rötter för olika funktioner ska också beräknas.

\section{Teori och lösning}

Sekantmetoden beräknar iterativt uppskattningar $x_n$ av en rot till en
funktion $f(x)$ enligt

\begin{equation}
    \label{eq:seq}
    x_{n+1} = x_{n} -
    \frac{f(x_n)}
        {\frac{f(x_n) - f(x_{n-1})}
                {x_n - x_{n-1}}
        }.
\end{equation}

Eftersom både $f(x_n)$ och $f(x_{n-1})$ används i beräkningen av $f(x_{n+1})$ krävs
två startlösningar, $x_0$ respektive $x_1$.

För att bestämma konvergensordningen $p$ används Ekvation~\ref{eq:order_of_convergence}, där
$C \geq 0$ och $x^*$ betecknar den exakta roten, då felet i skattning $x_{k+1}$ 
från den exakta roten antas vara mindre än felet i skattning $x_{k}$ upphöjt till
$p$ med konstant faktor.

\begin{equation}
    \begin{gathered}
        |x_{k+1} - x^*| \leq C |x_k - x^*|^p \Leftrightarrow \\
        \frac{|x_{k+1} - x^*|}{C} \leq |x_{k} - x^*|^p \Leftrightarrow \\
        \log\big(\frac{|x_{k+1} - x^*|}{C} \big) \leq
            p \cdot \log\big(|x_{k} - x^*|\big) \Leftrightarrow \\
        p \geq \frac{\log(\frac{|x_{k+1} - x^*|}{C})}
            {\log(|x_{k} - x^*|)}
        \approx \frac{\log(|x_{k+1} - x^*|)}
            {\log(|x_{k} - x^*|)}\\
    \end{gathered}
    \label{eq:order_of_convergence}
\end{equation}

$|x_n - x^*|$ bestämdes med \textit{metodoberoende feluppskattning} enligt 
Ekvation~\ref{eq:approx_error}, där $\xi$ ska ligga mellan $x_n$ och $x^*$. Eftersom
att de funktioner vi testar har en känd derivata används $\xi = x_n$

\begin{equation}
    |x_n - x^*| = \frac{|f(x_n)|}{|f'(\xi)|} = \frac{|f(x_n)|}{|f'(x_n)|}
    \label{eq:approx_error}
\end{equation}


\subsection{Beräkningar}

Testerna kördes på följande funktioner:

\begin{itemize}
    \item $x^2 - 2$
    \item tan(x)
    \item sin(x)
    \item $x^3 + 1$
    \item $x^2$
\end{itemize}

Parametrarna som användes och de exakta rötterna redovisas i Tabell~\ref{tab:params}.

\begin{table}[h]
    \centering
    \begin{tabular}{c | c | c | c | c | c}
        \textbf{Parameter}  & $\mathbf{x^2 - 2}$    & \textbf{tan(x)}   & \textbf{sin(x)}   & $\mathbf{x^3 + 1}$ & $\mathbf{x^2}$ \\ \hline
        $x_0$               & 1                     & -0.3                & -0.4                   & -1.1 & -0.1 \\
        $x_1$               & 2                     & 0.5                 & 0.5                  & -0.9 & 0.5 \\
        Exakt rot           & $\sqrt{2} $           & 0                  & 0                  & -1 & 0 \\
        
    \end{tabular}
    \caption{De parametrar som användes i beräkningarna}
\label{tab:params}
\end{table}


\section{Resultat}

I Tabell~\ref{tab:roots} ges approximationerna av rötterna till de olika
funktionerna som analyserats. I Tabell~\ref{tab:ps} ges en approximation av 
konvergensordningen mellan varje iteration beräknad enligt
Ekvation~\ref{eq:order_of_convergence}.

Funktionen $x^2$ med dubbelroten $x=0$ kördes också, men konvergerade så långsamt att det är 
opraktiskt att redovisa i en tabell. Om man använder startlösningarna $x_0 = -0.1$ och $x_1 = 0.05$ avslutade den efter 66 iterationer, med slutlösningen $1.609962437506356\cdot10^{-15}$. Konvergensordningen $p$ blev ungefär 1.014330.

\begin{table}[h]
    \centering
    \begin{tabular}{c | c | c | c | c}
        \textbf{n} & $\mathbf{x^2 - 2}$ & \textbf{tan(x)} & \textbf{sin(x)} & $\mathbf{x^3 + 1}$ \\ \hline
        1 & 1.333333333333333 & -1.077864024040920$\cdot10^{-2}$        &  3.382605065430477$\cdot10^{-3}$  & -0.990033222591362 \\
        2 & 1.400000000000000 & -8.955047317953449$\cdot10^{-4}$        & -1.461886022731283$\cdot10^{-4}$  & -1.001074308675966 \\
        3 & 1.414634146341463 & -3.756104606641650$\cdot10^{-8}$        &  2.667340781299481$\cdot10^{-10}$ & -0.999989228883491 \\
        4 & 1.414211438474870 & -1.004084892145777$\cdot10^{-14}$       & -9.500653752817749$\cdot10^{-19}$ & -0.999999988436696 \\
        5 & 1.414213562057320 & -4.733165431326071$\cdot10^{-30}$       & --                                & -1.000000000000125 \\
        6 & 1.414213562373095 & --                                      & -- & -1.000000000000000 \\
        
    \end{tabular}
    \caption{Approximation av roten till olika uttryck vid olika antal iterationer}
\label{tab:roots}
\end{table}

\begin{table}[h]
    \centering
    \begin{tabular}{c | c | c | c | c}
        \textbf{n} & $\mathbf{x^2 - 2}$ & \textbf{tan(x)} & \textbf{sin(x)} & $\mathbf{x^3 + 1}$ \\ \hline
        1 & 1.691420 & 1.549190 & 1.552196 & 1.483364 \\
        2 & 1.827615 & 2.436164 & 2.496403 & 1.673276 \\
        3 & 1.680278 & 1.885217 & 1.882431 & 1.597692 \\
        4 & 1.674748 & 2.094897 & --       & 1.625895 \\
        5 & 1.647635 & --       & --       & Inf \\
        
    \end{tabular}
    \label{tab:ps}
    \caption{$p \approx \log(|x_{n + 1} - x^*|)/\log(|x_n - x^*|)$ vid olika iterationer}
\end{table}

\section{Diskussion}

Konvergensordningen för Newton-Raphsons metod är 2, och eftersom sekantmetoden
inte använder den analytiska derivatan är det rimligt att dess konvergensordning är lägre
än Newton-Raphson. Konvergensordningen 1.6-1.7 som fås vid beräkning av $x^2-2$ och
$x^3 + 1$ är därför rimlig. Den beräknade konvergensordningen för $x^3 + 1$ vid
iteration 5 förklaras av att den uppskattade roten har nästintill inget fel i
iteration 6.

De trigonometriska funktionerna har högre och mer sporadisk konvergensordning,
något som vi är osäkra på varför. En möjlig förklaring är att derivatan här är
lägre än för $x^2-2$ och $x^3 + 1$, men vidare undersökning krävs för att
verifiera detta.

Sekantmetoden konvergerar väldigt långsamt för $x^2$, eftersom den har en
dubbelrot. Detta betyder att derivatan vid roten är 0, vilket leder till

\[
    \lim_{n \to \infty} f(x_n) - f(x_{n+1}) = 0,
\]

vilket gör att hoppen mellan varje iteration kommer att vara väldigt stora och att
approximationen oscilerar runt roten. Det ger en konvergensordning strax över
1, vilket gör att denna metod inte lämpar sig för uppskattning av dubbelrötter.

\section*{Bilagor}
\appendix

\section{MATLAB-kod för sekantmetoden}
\lstinputlisting{../sekant.m}

\section{MATLAB-kod för beräkning av fel och aritmetisk komplexitet}
\label{sec:testcode}

\lstinputlisting{../main.m}
\end{document}
