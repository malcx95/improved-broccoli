\documentclass[a4paper,titlepage]{article}
\usepackage[utf8]{inputenc} %Make sure all UTF8 characters work in the document
\usepackage{listings} %Add code sections
\usepackage{color}
\usepackage{graphicx}
\usepackage{titling}
\usepackage{textcomp}
\usepackage{amsmath}
\usepackage[hyphens]{url}
\usepackage[bottom]{footmisc}
\definecolor{listinggray}{gray}{0.9}
\definecolor{lbcolor}{rgb}{0.9,0.9,0.9}
\usepackage[framed,numbered,autolinebreaks,useliterate]{mcode}
% \lstset{	%ifall du ska koda.
% 		backgroundcolor=\color{lbcolor},
% 		tabsize=4,
% 		rulecolor=,
% 		upquote=true,
% 		aboveskip={1.5\baselineskip},
% 		columns=fixed,
% 		showstringspaces=false,
% 		extendedchars=true,
% 	    breaklines=true,
% 	    prebreak = \raisebox{0ex}[0ex][0ex]{\ensuremath{\hookleftarrow}},
% 	    frame=single,
% 	    showtabs=false,
% 	    showspaces=false,
% 	    showstringspaces=false,
% 	    identifierstyle=\ttfamily,
% 	    keywordstyle=\color[rgb]{0,0,1},
% 	    commentstyle=\color[rgb]{0.133,0.545,0.133},
% 	    stringstyle=\color[rgb]{0.627,0.126,0.941},
%         %language=matlab
% }

%Set page size
\usepackage{geometry}
\geometry{margin=3cm}
\usepackage{parskip} 
%\pretitle{%	en bild för framsidan
	%\begin{center}
	%\LARGE
%	\includegraphics[width=6cm]{python.png}\\[\bigskipamount]
%}
\title{
    \textbf{Projekt 1 i Beräkningsmatematik -- Integration }}
\date{\today}
\author{%
    Malcolm Vigren \\
    \texttt{malvi108@student.liu.se}
    \and
    Frans Skarman\\
    \texttt{frask812@student.liu.se}
    }
\renewcommand*\contentsname{Innehållsförteckning}
\renewcommand*\tablename{Tabell}

\begin{document}
\maketitle
\newpage
\tableofcontents
\newpage

\section{Introduktion}

\subsection{Uppgift}

Syftet med denna undersökningen är att implementera och utvärdera sekantmetoden
för lösning av ickelinjära ekvationer.


\section{Teori}

Sekantmetoden beräknar en iterativt uppskattningar $x_n$ av en rot till en
funktion $f(x)$ enligt ekvation~\ref{eq:seq}

\begin{equation}
    \label{eq:seq}
    x_{n+1} = x_{n}
    \frac{f(x_n)}
        {\frac{f(x_n) - f(x_{n_1})}
                {x_n - x_{n-1}}
        }
\end{equation}

Eftersom att både $f(x_n)$ och $f(x_{n-1})$ används i beräkningen av $f(x_{n+1})$ krävs
två stycken initiala gissningar.


För att bestämma konvergensordningen $p$ används ekvation~\ref{eq:order_of_convergence} där
$C \geq 0$.

\begin{equation}
    \begin{gathered}
        |x_{k+1} - x^*| \leq C |x_k - x^*|^p \Leftrightarrow \\
        \frac{|x_{k+1} - x^*|}{C} \leq |x_{k} - x^*|^p \Leftrightarrow \\
        \log\big(\frac{|x_{k+1} - x^*|}{C} \big) \leq
            p \cdot \log\big(|x_{k} - x^*|\big) \Leftrightarrow \\
        \frac{\log(\frac{|x_{k+1} - x^*|}{C})}
            {\log(|x_{k} - x^*|)}
        \approx \frac{\log(|x_{k+1} - x^*|)}
            {\log(|x_{k} - x^*|)} \leq p \\
    \end{gathered}
    \label{eq:order_of_convergence}
\end{equation}

$|x_n - x^*|$ bestämdes med ``metodoberoende feluppskattning'' enligt 
ekvation~\ref{eq:approx_error} där $\xi$ ska ligga mellan $x_n och x^*$. Eftersom
att de funktioner vi testar har en känd derivata används $\xi = x_n$

\begin{equation}
    |x_n - x^*| = \frac{|f(x_n)|}{|f'(\xi)} = \frac{|f(x_n)|}{|f'(x_n)}
    \label{eq:approx_error}
\end{equation}


För att beräkna ett värde på $p$ beräknades rötterna av funktionerna $x^2$, $\tan(x)$
och $\sin(x)$.



\subsection{Lösning}


\section{MATLAB-kod}

\lstinputlisting{../sekant.m}

\section{Resultat}

I tabell~\ref{tab:roots} ges approximationerna av rötterna till de olika
funktionerna som analyserats.



\begin{table}[h]
    \centering
    \label{tab:roots}
    \begin{tabular}{c | c | c | c}
        \textbf{Iteration} & $\mathbf{x^2 - 2}$ & \textbf{tan(x)} & \textbf{sin(x)} \\ \hline
        1 & 1.333333333333333 & -1.077864024040920$\cdot10^{-2}$       &  3.382605065430477$\cdot10^{-3}$ \\
        2 & 1.400000000000000 & -8.955047317953449$\cdot10^{-4}$       & -1.461886022731283$\cdot10^{-4}$ \\
        3 & 1.414634146341463 & -3.756104606641650$\cdot10^{-8}$       &  2.667340781299481$\cdot10^{-10}$ \\
        4 & 1.414211438474870 & -1.004084892145777$\cdot10^{-14}$    & -9.500653752817749$\cdot10^{-19}$ \\
        5 & 1.414213562057320 & -4.733165431326071$\cdot10^{-30}$    & -- \\
        6 & 1.414213562373095 & -- & -- \\
        
    \end{tabular}
    \caption{Den uppskattade roten vid olika iterationer}
\end{table}

\begin{table}[h]
    \centering
    \label{tab:ps}
    \begin{tabular}{c | c | c | c}
        \textbf{Iteration} & $\mathbf{x^2 - 2}$ & \textbf{tan(x)} & \textbf{sin(x)} \\ \hline
        1 & 1.691420 & 1.549190 & 1.552196 \\
        2 & 1.827615 & 2.436164 & 2.496403 \\
        3 & 1.680278 & 1.885217 & 1.882431 \\
        4 & 1.674748 & 2.094897 & -- \\
        5 & 1.647635 & --       & -- \\
        
    \end{tabular}
    \caption{$\log(|x_{n + 1} - x^*|)/\log(|x_n - x^*|)$ vid olika iterationer}
\end{table}

\section{Diskussion}



\section*{Bilagor}
\appendix

\section{MATLAB-kod för beräkning av fel och aritmetisk komplexitet}
\label{sec:testcode}
\end{document}
